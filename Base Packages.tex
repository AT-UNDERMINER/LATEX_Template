\usepackage{graphicx}  % For including images (allows resizing, rotating, and including images)
\usepackage{mwe}
\usepackage{amsmath}   % For math equations (provides advanced mathematical formatting)
\usepackage{amssymb}   % For additional math symbols (extends the symbol collection for math mode)
\usepackage{hyperref}  % For hyperlinks (enables clickable URLs, references, and links in the document)
\usepackage{lipsum}    % For generating placeholder text
\usepackage{caption}   % For customising captions (allows customisation of figure and table captions)
\usepackage{amsthm}    % For theorem environments (provides easy formatting for theorems, lemmas, etc.)
\usepackage{float}     % For precise figure placement (provides control over figure/table placement using [H] option)
\usepackage{xcolor}    % For coloured text (allows defining and using colours in text and backgrounds)
\usepackage{subcaption} % For subfigures (allows the inclusion of subfigures with separate captions in a single figure)
\usepackage{listings}  % For including code (formats and includes code snippets in various programming languages)
\usepackage{cite}      % For citing references (provides comprehensive management and formatting of citations)
\usepackage{multicol}  % For multi-column layout (allows text to be displayed in multiple columns)
\usepackage{parskip}   % To insert paragraph spacing
\usepackage{textcomp, gensymb}   % Add Extra symboles like degree
\usepackage{tikz}   % Allow for FBD Drawings
\usepackage{siunitx}    % Type setting or unit and unitless numbers
\usepackage{pgfplots}
\usepackage{circuitikz}
\usepackage{titlesec}
\usepackage{layout}
\pgfplotsset{compat=newest} 
